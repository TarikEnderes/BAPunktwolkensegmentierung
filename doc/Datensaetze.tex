\chapter{Datens�tze}
	\section{Cityscapes}
		Cityscapes ist ein �ffentlich zug�nglicher Datensatz f�r Segmentierung. Er bietet 5000 fein und 20000 grob auf Pixel-Ebene annotierte Bilder f�r semantische oder Instanzsegmentierung. Der Satz an fein annotierten Aufnahmen, der in den Experimenten verwendet wird, ist unterteilt in einen Trainingssatz aus 2975 Bildern, einem Evaluierungssatz von 500 Bildern und einem Testsatz aus 1525 Bildern. Aufgenommen ist der Datensatz von einem Auto aus in 50 gr��tenteils deutschen St�dten, jeweils am Tag bei sonnigem oder bew�lktem Wetter um Fr�hling, Sommer und Herbst. Die Bilder zeigen ausschlie�lich Szenen, die sich auf vielbefahrenen Stra�en abspielen.\\
		Die Daten sind aufgenommen mit einer Stereo-Kamera, die hinter der Windschutzscheibe des Fahrzeugs angebracht war, bei einer Frame-Rate von 17Hz. Die Bilder des Datensatzes sind kalibriert, Bayer-gefiltert und rektifiziert.
		\\F�r weitere Informationen siehe \cite{Cityscapes}.
	\section{KITTI}
		Das in \cite{KITTI} beschriebene KITTI ist ein Datensatz f�r Forschung in den Bereichen mobile Robotik und autonomes Fahren. Es werden darin Kamerabilder, Laserscans, GPS- und IMU-Daten zur Verf�gung gestellt. Die Kamerabilder werden von zwei Stereo-Kamera-Rigs aufgenommen, eines f�r Farbaufnahmen, eines f�r Graustufenbilder und liegen sowohl als Rohdaten als auch rektifiziert vor. Die Laserscan-Daten sind im Velodyne LiDAR-Format gespeichert. Die Kalibrierungs-Matrizen sind im Rohdatensatz ebenfalls angegeben.\\
		Der Datensatz umfasst insgesamt 6 Stunden an Aufnahmen mit zwischen 10Hz und 100Hz aus Karlsruhe. Die Messger�te sind auf einer mobilen Plattform auf einem Auto angebracht. Die Perspektive unterscheidet sich also geringf�gig von der der Cityscapes-Daten. F�r weitere Informationen �ber die verwendeten Messger�te siehe \cite{KITTI}.\\
		Im Gegensatz zum Cityscapes-Datensatz, der sich auf Stra�enverkehr spezialisiert, wird im KITTI-Datensatz versucht, eine m�glichst gro�e Szenenvielfalt anzubieten.