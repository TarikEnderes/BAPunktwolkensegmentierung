\chapter{Einleitung}
	F�r zahlreiche Entwicklungsthemen der heutigen Zeit, wie beispielsweise autonomes Fahren, ist eine pr�zise Erkennung der Umweltbedingungen unerl�sslich. Kameras und Laserscanner finden f�r diesen Zweck oft Verwendung, was die Verarbeitung von Bildern und Punktwolken zu einem verbreiteten Gegenstand moderner Forschung macht. H�ufig wird zur L�sung dieser komplexen Probleme auf Elemente der Neuroinformatik zur�ckgegriffen. \\
	Je nach Anwendungsfeld ist ein bestimmter Grad an Auswertung der gegebenen Daten erforderlich. Diese Arbeit befasst sich mit der Aufgabe, Bilder und Punktwolken zu segmentieren.
	\section{Segmentierung}
		Segmentierung bezeichnet einen Vorgang, bei dem ein Bild nach bestimmten Homogenit�tskriterien in inhaltlich zusammenh�ngende Regionen eingeteilt wird. Von den verschiedenen Ans�tzen, die das erreichen sollen, befasst sich diese Arbeit mit pixelbasierten Verfahren, bei denen jedem Pixel in einem Bild eine Klasse zugeordnet wird. Man unterscheidet zwischen den in \cite{UPSNet} beschriebenen, semantische Segmentierung, Instanz-Segmentierung und panoptische Segmentierung und der in \cite{DBLP:journals/corr/abs-1904-09172} ausgef�hrten Objekt-Segmentierung. F�r weitere Informationen siehe \cite{gonzalez2008digital}.
		\subsection{Semantische Segmentierung}
			Bei der semantischen Segmentierung soll jeder Pixel eine valide Klasse erhalten. Es wird dabei nicht zwischen unterschiedlichen Instanzen einer Objektklasse unterschieden. Wenn beispielsweise auf einem Bild zwei Fahrzeuge zu sehen sind und bei der Segmentierung die Klasse "`Fahrzeug"' zugeteilt werden soll, erhalten die Pixel beider Fahrzeuge das Label "`Fahrzeug"'. Die Anzahl valider Klassen bleibt somit bei jeden prozessierten Bild gleich.\\ Einige Anwendungsgebiete von semantischer Segmentierung sind autonomes Fahren im Gel�nde \cite{OffRoad}, Zellanalyse in der Biomedizin \cite{journals/corr/RonnebergerFB15} und Auswertung von Satellitenbildern f�r Kartographie \cite{DBLP:journals/corr/abs-1904-03983}.
		\subsection{Instanz-Segmentierung}
			Im Gegensatz zur semantischen Segmentierung werden bei der Instanz-Segmentierung nur z�hlbare Objekte betrachtet und deren Instanzen ber�cksichtigt. �bertragen auf vorheriges Beispiel w�rden die Pixel des einen Fahrzeug ein Label wie "`Fahrzeug1"' und die des anderen analog "`Fahrzeug2"' erhalten.\\
			Instanz-Segmentierung findet beispielsweise Anwendung zur Detektion von Personen in Videodaten f�r Verhaltensanalysen und �berwachung \cite{HumanSegmentation}.
		\subsection{Panoptische Segmentierung}
			Die panoptische Segmentierung stellt eine Kombination der vorherigen Segmentations-Arten dar. Z�hlbare Objekte werden demnach nach dem Prinzip der Instanz-Segmentierung und amorphe nach dem der semantischen Segmentierung segmentiert. Die Ergebnisse beider Verfahren werden anschlie�end kombiniert.
		\subsection{Objekt-Segmentierung}
			Bei der Objekt-Segmentierung soll f�r jeden Pixel eines Bildes entschieden werden, ob er Teil des Vorder- oder des Hintergrundes ist, weshalb sie h�ufig als Vordergrund-Hintergrund-Segmentierung bezeichnet wird. Von Interesse ist dabei nur, wo sich Objekte im Bild befinden, nicht, wie bei den anderen Disziplinen, worum es sich handelt. Oft ist das Ziel dabei die Erkennung von Bewegung in Videodaten.\\
			Objekt-Segmentierung findet Verwendung im Bereich der Video�berwachung \cite{ObjSegmentation}.
		
	\section{Ziele und Anforderunen}
		Ziel der Arbeit ist es, ein System zu entwickel, das mit Hilfe von neuronalen Netzen ein Bild semantisch segmentiert und aufgrund der so entstandene Labels auf Pixelebene eine Punktwolke derselben Szene segmentiert. Der Anwendungsbereich des Systems soll autonomes Fahren sein, weshalb Entwicklung und Experimente mit Datens�tzen f�r diesen durchgef�hrt werden. Konkret soll die Ausgabe des Systems Algorithmen f�r Einf�delvorg�nge an Kreuzungen verbessern. Besondere Wichtigkeit kommt daher der Erkennung von Fahrzeugen, Personen und Stra�en zu. Eine Kernanforderung ist dabei Echtzeitf�higkeit. Optimierung der Laufzeit ist also essentiell. Weiterhin soll das System transportabel, leicht zu verwenden, benutzerfreundlich und ressourcenschonend sein.\\
		Die Entwicklung erfolgt in Python mit CUDA-Unterst�tzung unter Verwendung des von Google entwickelten Framework DeepLab, das zur Zeit der Entstehung dieser Arbeit als State-of-the-Art angesehen wird.\\