\chapter{Zusammenfassung}
	Ziel der Arbeit war die Entwicklung eines Systems zur semantischen Segmentierung von Bildern und Punktwolken mit neuronalen Netzen. Es soll also jedem Pixel eines Bildes, beziehungsweise jedem Punkt einer Punktwolke eine Klasse zugeteilt werden, ohne zwischen Instanzen von z�hlbaren Objekten zu unterscheiden wie das bei Instanz- oder panoptischer Segmentierung der Fall w�re.\\
	Als Netzarchitektur wurde das von Google entwickelte DeepLab verwendet. Dabei handelt es sich um ein Deep Fully-Convolutional Neural Network, das durch das Adaptieren von Atrous Convolution, Atrous Spatial Pyramid Pooling und Conditional Random Fields auf den Bereich der semantischen Bildsegmentierung angepasst ist.\\
	Wegen der Anforderungen an die Laufzeit wurde als Backbone das leichtgewichtige MobileNetV2 gew�hlt, das in Experimenten mit dem leistungsf�higeren Xception65 verglichen wurde. Beide Netze sind als Residual Networks strukturiert, was bedeutet, dass �ber so genannte Rasidual Connections Daten �ber mehrere Verarbeitungsschichten hinweg unver�ndert weitergeleitet und auf die Ergebnisse der �bersprungenen Schichten addiert werden. Dadurch wird verhindert, dass ein Hinzuf�gen weiterer Schichten die Ergebnisse verschlechtert. Eine weitere Technik, die beide Architekturen verwenden ist Depthwise Separable Convolution, bei der zuerst eine r�umliche und dann eine dimensions�bergreifende Faltung durchgef�hrt wird. MobileNetV2 zeichnet sich durch die Verwendung von Bottleneck-Bl�cken aus. Darin werden die Daten zuerst expandiert, dann komprimiert, um Speicherplatz zu sparen.\\
	Da diese Arbeit das Ziel hat, Algorithmen f�r autonomes Fahren zu unterst�tzen und zu verbessern, wurde die Implementierung mit Hilfe der Datens�tze Cityscapes und KITTI ausgewertet, die explizit f�r diesen Bereich konzipiert sind. Dabei dienten die 5000 fein annotierten Bilder f�r Segmentierung von Cityscapes als haupts�chlicher Datensatz zu Training und Evaluierung des Netzes. Der KITTI-Datensatz enth�lt Laserscans, beziehungsweise Punktwolken und Bilder, die mit kalibrierten Kameras aufgenommen wurden, was es erm�glicht, die auf den Bildern erkannten Labels durch Ber�cksichtigung der Projektionsmatrix auf die zugeh�rigen Punktwolken zu projizieren.\\
	Die ideale Trainingsdauer von 25 Epochen wurde f�r ein Netz mit MobileNetV2 experimentell ermittelt. Die Ergebnisse der verschiedenen Models wurden daf�r in IoU-Metrik bewertet. Das Netzwerk erzielte dabei die besten Ergebnisse beim Erkennen amorpher Objekte, schlechtere beim Erkennen von Details im Bild. Der Vergleich mit einem Xception65-Model ergab, wie zu erwarten war, dass Xception bessere Ergebnisse bei l�ngerer Rechenzeit liefert. Als gr��tes Problem stellte sich Overfitting heraus, wie ein Versuch mit Verfeinerung mit Trainingsdaten aus dem KITTI-Datensatz zeigt. Bereits nach einer Trainingsepoche mit einem vergleichsweise kleinen Datensatz verschlechterten sich die Ergebnisse auf dem Cityscapes-Datensatz merklich. Es stellte sich als vorteilhaft heraus, die KITTI-Daten beim Lernprozess miteinzubeziehen und so die Variet�t in den Trainingsdaten zu erh�hen.\\
	Zuk�nftige Experimente k�nnten zum Ziel haben, die Hyper-Parameter wie Dropout- und Regularisierungs-Rate anzupassen. Auch das Hinzuf�gen und Entfernen von Verarbeitungsschichten im Netzwerk k�nnte eine M�glichkeit sein, die Ergebnisse zu verbessern. Zur Verbesserung der Laufzeit k�nnten Bilder mit unterschiedlicher Aufl�sung getestet werden.